\begin{abstract}
\addcontentsline{toc}{section}{摘要}		% 在目录中显示摘要

%========================================
活性物质是一类典型的非平衡态体系,作为软凝聚态物理学中新兴的一个研究方向,在最近二十年来变得非常活跃。机器学习经过七十多年的发展,取得了巨大的进步,正在越来越多的领域发挥着重要作用。机器学习作为一种数据驱动的技术,对活性物质研究中的大量数据是否有效,这是一个值得探究的问题。本文通过进行了机器学习识别和分类活性物质图像、高维数据可视化、机器学习对活性物质图像进行聚类分析和以活性物质图像检索活性物质图像等几个机器学习处理活性物质研究数据的数值实验发现,机器学习在活性物质图像数据的识别和分类、检索、聚类以及高维数据可视化等方面具有强大的能力。\\
\textbf{关键词}: \ \ 活性物质;机器学习;图像识别和分类;图像检索;图像聚类;数据可视化
\end{abstract}

%========================================
% 英文摘要
\renewcommand\abstractname{Abstract}
\begin{abstract}
As a new research direction in soft condensed matter physics, active matter, a typical out-of-Equilibrium system, have become very active in the last twenty years. After more than 70 years of development, machine learning has made great progress and is playing an important role in more and more fields. As a data-driven technology, whether machine learning is effective for large amounts of data in active matter research, is a question worth exploring. In this thesis, we carried out several numerical experiments of using machine learning to process active matter research data. We found that machine learning has a strong ability in the recognition and classification, retrieval, clustering and visualization of high-dimensional data of image data generated by active matter research.\\
\textbf{keywords}: \ \ Active matter; Machine learning; Images recognization and classifying; Images retrieval; Images clustering; Data visualization
\end{abstract}
