\section{前言}
活性物质是一类典型的非平衡态体系,已成为软凝聚态物理新近发展的一个重要研究方向。活性粒子是构成活性物质的基本单元,能够在某些自由度上以更强的动能进行运动,从而能够在一定程度上进行自驱动。这一特点导致活性粒子并不满足平衡态时的能量均分定理,从而使系统处于非平衡态\cite{shi2012}。我们可以理解为,活性粒子能够主动从周围环境中吸取能量来进行机械运动\cite{MLAM},就如蜂群中的每只蜜蜂独立地采蜜一样。每个粒子独立地进行运动,导致的结果就是系统的复杂性指数上升,而复杂性往往能够带来丰富的动力学现象,生命系统就是这方面的一个例子。

物理学是一门以自然现象为研究对象的学科,生命系统作为最复杂的自然现象之一,正日益受到物理学家们的关注和研究。人们发现,活性物质体系中的非平衡现象广泛存在于自然界,尤其是对生命现象具有非凡的意义。因此,对活性物质体系的研究,为我们从物理学的角度去理解生命有着非凡的意义\cite{shi2012}。

活性物质的研究过程中常常产生大量数据,机器学习作为一种数据驱动的技术,对活性物质研究中的大量数据是否有效,这是一个值得探究的问题。